\documentclass[11pt, a4paper, oneside]{book}
\usepackage{titling}
\title{Heading 1}
\date{2015-2016}
\author{Lorem Ipsum Author}
\usepackage{helvet} 				%Helvetica Font
\usepackage{lipsum}				% Om Lorem Ipsum te maken
\usepackage{tabularx}			% Make tables
\usepackage{enumitem}			% Make Lists
\usepackage[dutch]{babel}		% Nederlandse tekenafstand - woordafbraak
\usepackage{mdframed}			% Make text boxes
\usepackage[usenames,dvipsnames]{xcolor}
\usepackage{color}				% Kleuren definieren met variabelen
\usepackage{eurosym}			% Use the euro symbol in LaTeX

% Om afbeeldingen te importeren

	\usepackage{graphicx}
	\graphicspath{ {images/} }

% Kleuren

	% Later systeem bedenken om ook CMYK colors te doen.
	\definecolor{grijs}{RGB}{150,150,150}
	\definecolor{oranjerood}{RGB}{240,76,37}
	\definecolor{oranjerood50pc}{RGB}{248,163,127}
	\definecolor{blauwgroen}{RGB}{0,156,171}
	\definecolor{blauwgroen50pc}{RGB}{111,195,206}
% Begin The document
\begin{document}
% VOORPAGINA
	\begin{titlepage}
				\pagenumbering{gobble}
				\vspace*{130mm}
				\begin{flushright}
				\bf
				\color{oranjerood}
				{
					\scshape	
					\huge
					\fontfamily{phv}
					\selectfont
					\thetitle
				}
				
				\vspace{0mm}
				\LARGE
				Heading 2
				\end{flushright}
				\vspace{\fill}
				{\color{blauwgroen} Heading 3}
				
				\newpage

	\end{titlepage}
% INHOUDSTAFEL

\tableofcontents

\newpage

\pagenumbering{arabic}

% BEGIN


\chapter{Inleiding}

Dit is een LaTeX template voor een Thomas More hogeschool paper.

\newpage

\chapter{Winst}

	\begin{mdframed}[backgroundcolor=grijs!40,shadow=false,roundcorner=8pt]
		$$test = test - test$$
	\end{mdframed}
\end{document}
